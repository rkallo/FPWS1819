\section{Diskussion}
\label{sec:Diskussion}
Das Ziel des Versuches Faraday-Effekt war die effektive Masse von Elektronen in Halbleitern bzw. in Galliumarsenid zu ermitteln. Die Aufgaben dieses Versuchs bestanden darin die Messdaten vom B-Feld gegen z aufzutragen und daraus das maximale B-Feld abzulesen und am Ende mit Hilfe von der ermittelten Differenz der Drehwinkel die effektive Masse zu bestimmen. In Tabelle \ref{tab:diskussion} wird das Ergebnis für den experimentellen Wert der effektiven Masse aufgelistet sowie die Literaturangaben für Galliumarsenid\cite{EffectiveMass}. 

\begin{table}[htpb]
	\centering
	\caption{Berechnete effektive Elektronenmasse $m^*$, berechnetes Verhältnis $m^*/m_{\text{e}}$ und Vergleich mit Literaturangaben\cite{EffectiveMass}.}
	\label{tab:diskussion}
	\begin{tabular}{cc}
		\toprule
		& schwach dotiert \\
		\midrule
	    $m^* / \SI{E-31}{\kilogram}$ & 1,055 $\pm$ 0,045\\
	    $\frac{m^*}{m_\text{e}}$ &  0,115 $\pm$ 0,0049\\
	    $\frac{m^*}{m_\text{lit}}$ & 0,067 \\
	    Abweichung $/ \si{\percent}$ &  41,73 \\
		\bottomrule
	\end{tabular}
\end{table}

Der Literaturwert für die effektive Elektronenmasse pro Elektronenmasse liegt bei $0,067$ und das experimentelle Ergebnis weicht davon um $\SI{41,73}{\percent}$ ab. Dies ist eine große Abweichung und für die Entstehung dieser Abweichung werden unten mögliche Fehlerquellen beschrieben.

Aufgrund des Versuchsaufbaus ergeben sich diverse Fehlerquellen: 
Bei der Messung ergab sich ein inhomogenes Magnetfeld im Bereich der Probenposition, wobei das Maximum des bestimmten Feldes leicht abseits der ungefähren Probenposition lag.

Die Nullabstimmung am Oszilloskop während der Winkelbestimmung erwies sich in vielen Fällen als schwierig, da das Minimum in einem kleinen Winkelintervall mit folglich steilen Flanken lag.
Da das Oszilloskop bei niedrigen Signalen einige Zeit gebraucht hat, um sich nachzujustieren, war es äußerst kompliziert, das tatsächliche Minimum zu bestimmen. Die Ursache für die Abweichung ist sicherlich der unsensible Minimal Abgleich am Oszilloskop. Das recht stark schwankende Signal erlaubte kaum eine exakte Bestimmung der Drehwinkel. Die Ursache für das stark schwankende Signal könnte in Frequenzschwankungen des Lichtszerhackers liegen. Zum anderen liegt noch ein Messfehler beim Ablesen der Winkel am Goniometer.
Diese Ungenauigkeit zeigt sich in der Abbildung \ref{fig:effektivemass1}, da einige Messwerte vom linearen Zusammenhang zwischen $\theta$ und $\lambda^2$ abweichen, zum Teil so stark, dass sie aus der Ausgleichsrechnung ausgeschlossen werden müssen. Auch in der Abbildung \ref{fig:messdaten} ist zu sehen, dass es keinen linearen Zusammenhang bei den Messwerten ergeben hat. 


Prinzipiell ist anzunehmen, dass die optischen Elemente nicht optimal justiert worden sind, was weitere Fehler in den Daten erzeugt.
Auch Intensitätsschwankungen der Lampe lassen sich als Fehlerquelle nicht ausschließen. In der Auswertung wurden Dispersionseffekte im Brechungsindex vernachlässigt.
