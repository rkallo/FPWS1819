\section{Diskussion}
\label{sec:Diskussion}
In Tabelle \ref{tab:exptheo} sind vergleichend Werte für die Debye-Temperaturen aus Literatur \cite{DebyeTemperaturKupfer}, theoretisch berechneten und aus Messwerten berechneten aufgeführt und die prozentuale Abweichung zwischen experimentell bestimmten Werten und den Werten aus der Literatur.

\begin{table}[htpb]
	\centering
	\caption{Vergleich zwischen Werten für die Debye-Temperaturen aus Literatur, theoretisch berechneten und aus Messwerten berechneten. Die Abweichung zwischen Literaturwert und experimentell bestimmten in Prozent.}
	\label{tab:exptheo}
	\begin{tabular}{rrrrr}
		\toprule
		$\theta_\text{D,lit}/\si{\kelvin}$ & $\theta_\text{D,exp}/\si{\kelvin}$ & Abweichung $/\si{\percent}$ & $\theta_\text{D,theo}/\si{\kelvin}$ & Abweichung $/\si{\percent}$ \\
		\hline
		345 & $354,79 \pm 88,02$ & 2,54 & 332,18 & 6,37 \\
		\bottomrule
	\end{tabular}
\end{table}

Der Literaturwert für die Debye-Temperatur von Kupfer liegt bei $\SI{345}{\kelvin}$. Das experimentelle Ergebnis weicht davon zu $\SI{2,54}{\percent}$, wohingegen der theoretisch bestimmte Werte nur um $\SI{6,37}{\percent}$ abweicht. 

Es lassen sich viele Gründe für die möglichen Abweichungen finden. Zuerst gibt es in der Tabelle \ref{tab:messwerte} einen kleinen Temperaturdifferenz zwischen der Temperatur der Probe und des Kupfer-Zylinders. Die Temperaturunterschiede zwischen den Messungen hätten bei $\SI{7}{\kelvin}$ bis $\SI{11}{\kelvin}$ liegen müssen. Diese haben dadurch ergeben, dass es schwierig war, ein konstant gleiches Erwärmen des Kupfer-Zylinders und der Probe zu gewährleisten, war daran lag, dass das Aufheizen des Kupfer-Zylinders langsam verlief. Zudem werden die Spannungs- und Stromgeräte manuell verstellt, wodurch der Zylinder und die Probe in unregelmäßigen Zeitintervallen erhitzt wurden. Außerdem ergeben sich noch andere Fehler wie z.B. das Vergessen die Zeit per Stoppuhr zu starten und mit Stickstoff weiter zu füllen. Dadurch, dass es nur kleine Unterschiede gaben, wurde den Wert für die Molwärme $C_V$ sehr groß(s. Tabelle \ref{fig:cvt2}). Die dazu berechneten Werte für die Molwärme sind auch in der Abbildung \ref{fig:cvt2} ersichtlich. In dem Plot der Molwärme gegen die Temperatur ist es auch zu sehen, dass die erste Hälfte der Messung durch diese kleinen Temperaturunterschiede große Werte für die Molwärme ergeben hat. Es lässt sich nicht erkennen, dass die berechneten Werte dem universellen Wert $3R$ entsprechen, der aus dem klassischen und dem Einstein-Modell folgt. 

Trotz der vielen Fehlerquellen hat sich letztendlich einen guten experimentellen Wert für die Debye-Temperatur ergeben. Es wurden nur bestimmte Werte aus der Abbildung \ref{fig:debyetemperatur} abgelesen, da für größere Werte als $> 24,9$ die zugehörigen $\frac{\theta_\text{D}}{T}$ nicht mehr abgelesen werden konnten. Hier sollte ja nur einen bestimmten Bereich für die Temperaturen betrachtet werden und zwar von $\SI{80}{\kelvin}$ bis $\SI{170}{\kelvin}$. Deshalb haben sich in der Tabelle \ref{tab:thetad} nur 7 möglichen Werte ergeben, die weiterhin in die Abbildung \ref{fig:debye} gegen die Temperatur aufgetragen wurden. Hier ist auch nicht zu erkennen, dass die Werte für die Debye-Temperatur auf einer geraden parallel zur x-Achse liegen. Ein Vorschlag wäre z.B. den ganzen Bereich zu betrachten, also die Temperaturen von $\SI{80}{\kelvin}$ bis $\SI{300}{\kelvin}$ um eine bessere Aussage für die Debye-Temperatur zu treffen.

Außerdem ist der Messaufbau nicht komplett gegen äußere Einflüsse abgeschirmt, da das Dewargefäß nach oben offen war und so ein Austausch der Luft stattfinden konnte, die die Probe umgibt, was zu einer schnelleren Erwärmung der Probe führt. Weitere Fehlerquellen, die bei der Berechnung nicht berücksichtigt wurden, sind Ungenauigkeiten der Widerstandsmessung, über die sich dann die Temperatur berechnet. 

Abschließend lässt sich sagen, dass der Versuch zur Bestimmung der Molwärme von Kupfer geeignet ist und sollte jedoch verstärkt darauf geachtet werden, dass das Gehäuse und die Probe etwa dieselbe Temperatur haben, damit es nicht zu größeren Abweichungen kommt.